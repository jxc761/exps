\documentclass[11pt]{article} % use larger type; default would be 10pt

\usepackage[utf8]{inputenc} % set input encoding (not needed with XeLaTeX)

%%% Examples of Article customizations
% These packages are optional, depending whether you want the features they provide.
% See the LaTeX Companion or other references for full information.

%%% PAGE DIMENSIONS
\usepackage{geometry} % to change the page dimensions
\geometry{a4paper} % or letterpaper (US) or a5paper or....
% \geometry{margin=2in} % for example, change the margins to 2 inches all round
% \geometry{landscape} % set up the page for landscape
%   read geometry.pdf for detailed page layout information

\usepackage{graphicx} % support the \includegraphics command and options

% \usepackage[parfill]{parskip} % Activate to begin paragraphs with an empty line rather than an indent

%%% PACKAGES
\usepackage{booktabs} % for much better looking tables
\usepackage{array} % for better arrays (eg matrices) in maths
\usepackage{paralist} % very flexible & customisable lists (eg. enumerate/itemize, etc.)
\usepackage{verbatim} % adds environment for commenting out blocks of text & for better verbatim
\usepackage{subfig} % make it possible to include more than one captioned figure/table in a single float
% These packages are all incorporated in the memoir class to one degree or another...

%%% HEADERS & FOOTERS
\usepackage{fancyhdr} % This should be set AFTER setting up the page geometry
\pagestyle{fancy} % options: empty , plain , fancy
\renewcommand{\headrulewidth}{0pt} % customise the layout...
\lhead{}\chead{}\rhead{}
\lfoot{}\cfoot{\thepage}\rfoot{}

%%% SECTION TITLE APPEARANCE
\usepackage{sectsty}
\allsectionsfont{\sffamily\mdseries\upshape} % (See the fntguide.pdf for font help)
% (This matches ConTeXt defaults)

%%% ToC (table of contents) APPEARANCE
\usepackage[nottoc,notlof,notlot]{tocbibind} % Put the bibliography in the ToC
\usepackage[titles,subfigure]{tocloft} % Alter the style of the Table of Contents
\renewcommand{\cftsecfont}{\rmfamily\mdseries\upshape}
\renewcommand{\cftsecpagefont}{\rmfamily\mdseries\upshape} % No bold!

%%% END Article customizations

%%% The "real" document content comes below...

\title{Facelet Representation for 3d Surface}
\author{Jing Chen}
%\date{} % Activate to display a given date or no date (if empty),
         % otherwise the current date is printed 

\begin{document}
\maketitle


As I said in our meeting, the surface is still defined in a viewer-centered way here. It is determined by its distance to camera. 


That means we describe the 3D surface in the camera coordinate system. The following figure illustrate the meaning of camera coordinate and camera plane. 

Mathematically, the camera coordinate system is aligned with the ``axes'' of the camera(target, up and right). Specifically, the origin of this system is at the camera position, y-axis directs the up of the camera, z-axis points from the camera to the focus and y-axis is orthogonal to xz plane. The camera plane means xy plane in this coordinate system. So the surface in the view is a function, $s(x, y)$ defined on xy plane. 

Facelets, ${f_i}$, are a set of functions on the xy plane. They code any surface $s(x, y)$ as follows.
	s(x, y)= \max{ c_i \f_i(x- \delta x_i, y - \delta y_i) }
 
The basic idea is illustrated in the following figure. Let's say $S$  is the surface we want to represent and $F_i$ and $F_j$ are two facelets. 
F(x, y)

The motivation of using min operation rather than a linear combination of basis functions is that the physical meaning of facelets is a piece of surface. If F_i less than F_j at (x, y) may mean the F_i is in the front of F_j. To combine them together, we should do "min" operation rather than to do "add" operation. The motivate of using \delta x_i is similar to the Spike Coding. 

Here we need to address two problems: (1) learning(determine the optimal kernel functions,i.e. facelets) (2) encoding (determining the optimal values of c_i, x_i, y_i).

To simplify the model, we use the soft maximum function to approximate the "min" operation.

S(x, y) = \max{ c_i  \f_i(x- \delta x_i, y - \delta y_i) }
				\approx \log( \sum \exp ( c_i  \f_i(x- \delta x_i, y - \delta y_i) ) }
				= \log( \sum s_i \exp ( f_i(x- \delta x_i, y - \delta y_i) ) }


To learning the basis function, we are trying to solve this optimization problem:

E = <S - \log( \sum s_i \exp ( F_i(x- \delta x_i, y - \delta y_i) ) } > + sp(a_i)

\end{document}










